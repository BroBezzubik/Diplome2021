\chapter*{ЗАКЛЮЧЕНИЕ}
\addcontentsline{toc}{chapter}{ЗАКЛЮЧЕНИЕ}
По итогу выполнения работы была спроектирована архитектура, разработано и протестировано программное обеспечение, осуществляющие извлечение ключевых слов из документов на русском языке.
Помимо выше сказанного были проведены исследования, продемонстрировавшие пригодность метода по работе с документами на кириллице.

В процессе выполнения работы:
\begin{enumerate}
	\item проанализирована предметная область и произведена классификация существующих методов по извлечению КС;
	\item разработана архитектура программного обеспечения;
	\item проведена модификация метода и разработаны модули;
	\item произведено исследование получившегося ПО.
\end{enumerate}

Из достоинств полученного метода можно выделить следующее:
\begin{enumerate}
	\item не требует обучения и наличия корпуса текстов;
	\item возможность извлечения n-компонентных ключевых слов;
	\item не использует тезаурусы. 
\end{enumerate}

Из возможных путей развития стоит отметить:
\begin{enumerate}
	\item добавление в предварительную обработку процесса стемминга;
	\item добавить авто определение языка;
	\item обучить алгоритм определять синонимические термины.
\end{enumerate}