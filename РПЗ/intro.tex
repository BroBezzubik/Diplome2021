\chapter*{ВВЕДЕНИЕ}
\addcontentsline{toc}{chapter}{ВВЕДЕНИЕ}
В истории развития цивилизации выделяют несколько \textbf{информационных революций}.
Первая была связана с изобретением письменности, благодаря чему появилась передача знаний между людьми без непосредственного общения.
Вторая (17-18 вв.) была вызвана благодаря изобретению книгопечатания и реформами школьной системы, сделавшими возможным массовое образование и популяризацию знаний и сыгравшими важную роль в становлении индустриального общества.
Третья и.р. (конец 19 - начало 20 вв.) началась с появлением телеграфа, телефона, радио, позволивших оперативно передавать информацию на любые расстояния и вступила в новую фазу с изобретением телевидения и компьютеров, а затем (кон. 1979-х гг.) с широким распространением информационных технологий. \cite{16}

Количество знаний, накопленных цивилизацией, поражает и, благодаря продолжающейся информационной революции, стремительно растет.
Традиционные медиа, такие как газеты и телевидение, мигрируют в интернет.
Газеты и другие СМИ обновляют новостые ленты в реальном времени, что позволяет интересующимся получать свежую информацию.
Ежедневно милионы пользователей интернета пораждают колосальное количество контента для различных целей, по разным вопросам, в разных странах, на всевозможных языках и в многообразных онлайн средах.
Это пользовательский контент в блогах и на сайтах социальных сетей, электронная почта, новости, научные работы и т.д.
По всему миру государства, институты, библиотеки, музеи цифвровизируют свои материалы и выкладывают их во всемирную сеть, что бы информацию для бизнеса, науки, исследований, развлечений можно было получить через любое доступное нам "умное" устройство: телефон, планшет, компьютер и т.д. \cite{2}

Таким образом, интернет и стал наиболее эффективными ресурсом для исследования современной экономики, культуры, политики, человеческого общения и взаимодействия людей. \cite{2}

Объемы циркулирующей в мировых телекоммуникационных сетях и хранящейся на серверах информации демонстрируют динамику взрывного роста.
На сегодняшний день, количество опубликованных документов достигает 1 биллиона веб-страниц.
Все указанное обуславливает увеличение состава и сложности программных решений в области обработки текстов на естественных языках в основе которых лежит ряд базовых алгоритмов, в том числе выделения или извлечения ключевых слов (КС).

Извлечение ключевых слов (Keyword extraction) - это задача по автоматическому определению набора терминов, которые наилучшем образом описывают "основную мысль" документа.
При изучении терминов, представляющих наиболее релевантную информацию, содержащуюся в документе, используется различная терминология: ключевые фразы, ключевые сегменты, ключевые термины, или просто ключевые слова.
Однако, несмотря на достаточно большое количество исследований в данной направлении, автоматическое извлечение ключевых слов представляет собой проблему, которая до сих пор не решена \cite{9}. 
Большинство существующих методов, успешно справляющихся с языками с бедной морфологией, такими как английский язык, не пригодны для естественных языков с богатой, в частности для русского языка, где каждая лексема характеризуется большим количеством словоформ с низкой частотностью в каждом конкретном тексте.

Методы способные на извлечения КС из русского языка, представляют в основном своем большинстве объемное программное обеспечение, требующее предварительного сбора и обработки корпуса текстов,  относящихся к одной предметной области, что влечет за собой узконаправленность методов, что ограничивает область применения.

Целью данной работы является разработка метода для автоматического извлечения ключевых слов из электронных документов на русском языке, способного работать с одиночным документом и не зависимого от его предметной области.

Для достижения поставленной цели необходимо выполнить следующие задачи:
\begin{enumerate}
	\item проанализировать современные методы извлечения ключевых слов;
	\item отобрать метод, который будет использоваться в качетсве базового в данной работе и предложить направления его модификации;
	\item описать основные шаги разработанного метода;
	\item разработать программное обеспечение для реализации предложенного метода;
	\item провести эксперементальное сравнение оригинального и модифицированного метода.
\end{enumerate}