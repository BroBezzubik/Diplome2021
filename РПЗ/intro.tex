\section{Введение}
В истории развития цивилизации выделяют несколько \textbf{информационных революций}.
Первая была связана с изобретением писмьменности, благодаря чему появилась передача знаний между людьми без непосредственного общения.
Вторая (17-18 вв.) была вызвана благодаря изобретению книгопечатания и реформами школьной системы, сделавшими возмоможным массовое образование и популиризацию знаний и сыгравшими важную роль в становлении индустриального общества.
Третья и.р. (конец 19 - начало 20 вв.) началась с появлением телеграфа, телефона, радио, позволивших оперативно передовать информацию на любые растояния и вступила в новую фазу с изобритением телевидения и компьютеров, а затем (кон. 1979-х гг.) с широким распространением информационных технологий. \cite{16}

Количество знаний, накопленных цивилизацией, поражает и, благодаря продолжающийся информационной революции, продолжает стремительно расти.
Традиционные медиа такие как газеты и телевидение мигрируют в интернет.
Новостные газеты или другие СМИ обновляют новостые ленты в реальном времени, что позволяет интересующимся получать свежую информацию.
Ежедневно милионы пользователей интернета пораждают колосальное количество контента для различных целей, по разным вопросам, в разных странах, на всевозможных языках и в многообразных онлайн средах.
Это пользовательский контент в блогах и на сайтах социальных сетей, электронная почта, новости, научные работы и т.д.
По всему миру государства, институты, библиотеки, музеи цифвровизируют свою материалы и выкладывают их в семирную сеть, что бы информацию для бизнеса, науки, исследований, развлечений можно было получить через любое доступное нам умное устройство: телефон, планшет, компьютер и т.д. \cite{2}

Таким образом, интернет и веб контент стали наиболее эффективными ресурсами для исследования современной экономики, культуры, политики, человеческого общения и взаимодействия людей. \cite{2}

На сегодняшний день, количество опубликованных документов достигает 1 биллиона веб-страниц \cite{1}. 
Такое очень огромное колличество информации делает задачу индексирования и поиска достаточно затруднительной, тем более преобладающие большинство документов не имеет ключевых слов (выражений) отсутствие которых заставляет пользователя полностью прочитать документ что бы получить общее представление о информации.
Проставлять в ручную ключевую информацию для текста быстро превращается в раздражающую задачу. 
При таком огромном количестве документов ручное проставление является невозможным. 
Для того что бы автоматизировать данный процесс часто используются программы для извлечения ключевых слов, которые используется для поиска ключевой идеи текста и извлечения/создания ключевых слов текста
Обычно результат данной работы представляет из себя от 5 — 15 ключевых значений, которые предстваляют информацию пользователю или специальным машинам общую информацию о документе.