\section{Введение}
В 21 веке текстовая ткань современного общества претерпело радикальные изменения в связи с продолжающейся информационной революцией.
Количество документов, доступных в Интернете и в других место, ошеломляет.
Люди, предприятия, группы, организации, учреждения и правительство не только оставляет "цифровые следы" при использовании Интернета.
Миллионы пользователей интернета, профессионалов или любителей создают миллиарды веб-страниц и документов.
Каждый день создается огромное количество онлайн-текстов для различных целей, по разным вопросам, в разных странах, на всевозможных языках и в многообразных онлайн средах: пользовательский контент в блогах и на сайтах социальных сетей, электронная почта, блоги, новости, научные работы и т.д.
Более того, по всему миру государства, институты, библиотеки, музеи цифвровизируют свою материалы и выкладывают его всемирную паутину, что бы информацию для бизнеса, науки, исследований, развлечений можно было получить через любое доступное нам устройство: телефон, планшет, компьютер и т.д. \cite{2}

Традиционные медиа такие как газеты и телевидение быстро мигрируют в интернет.
Новостные газеты или другие СМИ обновляют новостые ленты почти в реальном времени, что позволяет интересующимся получать свежую информацию.
Поисковые системы только усугубили ситуацию, делая все больше и больше документов доступными всего в несколько нажатий клавиш на вашей клавиатуре.
Таким образом, интернет и веб контент стали наиболее эффективными ресурсами для исследования современной экономики, культуры, политики, человеческого общения и взаимодействия людей. \cite{2}

На сегодняшний день, количество опубликованных документов достигает 1 биллиона веб-страниц \cite{1}. 
Такое очень огромное колличество информации делает задачу индексирования и поиска достаточно затруднительной, тем более преобладающие большинство документов не имеет ключевых слов (выражений) отсутствие которых заставляет пользователя полностью прочитать документ что бы получить общее представление о информации.
Проставлять в ручную ключевую информацию для текста быстро превращается в раздражающую задачу. 
При таком огромном количестве документов ручное проставление является невозможным. 
Для того что бы автоматизировать данный процесс часто используются программы для извлечения ключевых слов, которые используется для поиска ключевой идее текста и извлечения/создания ключевых слов текста
Обычно результат данной работы представляет из себя от 5 — 15 ключевых значений, которые предстваляют информацию пользователю или специальным машинам общую информацию о документе.

Целью данной работы является разработка метода извлечения ключевых словосочетаний или слов из текста электронных документов.
Для достижения поставленной выше цели необходима решить следующие задачи:

\begin{enumerate}
	\item Анализ темы и предметной области
	\item Изучить существующие методы решения поставленной цели
	\item Реализовать алгоритмы для извлечения ключевых слов.
	\item Тестирование и замер результатов реализаций
	\item Анализ полученных результатов и сопоставление их друг с другом
	\item Вывод по итогам проекта
\end{enumerate}