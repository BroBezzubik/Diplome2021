\section{Заключение}
По итогу выполненния работы была спроектирована архитектура, разработано и протестировано программное обеспечение, осуществляющие извлечение ключевых слов из документов на русском языке.
Помимо выше сказаного были проведены иследования, продемострировашие пригодность метода по работе с документами на русском языке.

В процессе выполнения работы:
\begin{enumerate}
	\item проанализирована предметная область и произведена классификация существующих методов КС;
	\item разработана архитектура программного обеспечения;
	\item проведена модификация метода и разработы модули;
	\item произведено исследование получивашегося ПО.
\end{enumerate}

Из достоинства полученного метода можно выделить следующее:
\begin{enumerate}
	\item не требует обучение;
	\item не требует наличие корпуса текстов;
	\item возможность извлечения n-компонентных ключевых слов;
	\item не использует тезаурусы. 
\end{enumerate}

Из возможных путей развития стоит отметить:
\begin{enumerate}
	\item добавление в предварительную обработку процесса стемминга;
	\item добавить авто авпределение языка;
	\item обучить алгоритм определять синонимические термины.
\end{enumerate}