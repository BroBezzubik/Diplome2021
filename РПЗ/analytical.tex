\section{Аналитический раздел}
В данном разделе описана предметная область. Указана классификация методов извлечения ключевых слов и

\subsection{Предметная область}
Перед тем как углубимся в предметную область стоит разобраться разницей между Анализом текста (Text Analysis), Интеллектуальным анализом текста (Text mining) и Аналитикой текста (Text Analytics). Анализ текста и Интелектуальный анализ текста это одно и тоже, они синонимичны друг другу.

TODO Дописать сравнение

\subsubsection{Что такое Интеллектуальный анализ текста?}
Интеллектуальный анализ текста - это процесс преобразования неструктурированной, сырой информации в структурированный формат для выявления значимых закономерностей и новых идей.\cite{6}
Применяя передовые аналитические методы, такие как метод наивного Байеса, метод опорных векторов (SVM) и другие алгоритмы глубокого обучения, компании могут исследовать и обнаруживать скрытые взаимосвязи в своих неструктурированных данных.

Тест - это один из самых распространенных типов данных в базах данных.
В зависимости от базы данных, данные могут быть организованы как:
\begin{enumerate}
	\item структурированные данные: это данные предствалены в табличном формате с многочисленными строками и столбацами, что упрощает их хранение и обработку для анализа и алгоритмов машинного обучения;
	\item не структурированные данные: эти данные не имеют предопределенного формата данных. Он может включать текст из источников, таких как социальные сети или обзоры продуктов, или мультимедийные форматы, такие как видео и аудиофайлы;
	\item полуструктурированные данные: как следует из названия эти данные предстваляют собой смесь форматов структурированных и неструктурированных данных. Хотя у него есть некторая организация, у него недостаточно структуры для удовлетворения требований реляционной базы данных. Примера частично структурированных данных включают файл XML, JSON, HTML.
\end{enumerate}

Так как 80\% иформации в мере относятся к неструктурированному формуту, интелектуальный анализ текста является черезвычайно выжным.

\subsubsection{Токенизация} 


\subsection{Классификация методов извлечения ключевых слов}


