\addcontentsline{toc}{section}{Список использованных источников}
%\begingroup
\makeatletter \renewcommand\@biblabel[1]{#1.} \makeatother
\renewcommand\refname{Список использованных источников}
\begin{thebibliography}{00}
	\bibliographystyle{ugost2008}
	\bibitem{1}
	Появился рейтинг стран по длине автодорог. Мы в пятерке. [Электронный ресурс] -- URL: $https://www.zr.ru/content/news/916887-rejting-stran-po-dline-avtomob/$ (Дата обращения: 24.12.2020)
	\bibitem{2}
	Сёмина В.А. Классификация методов математического моделирования транспортных потоков. [Электронный ресурс] $//$ Политехнический молодёжный журнал МГТУ им. Н.Э.Баумана. -- 2018. -- Номер 2(19). -- С.5. -- URL: DOI $10.18698/2541-8009-2018-2-247$ (Дата обращения: 06.12.2021)
	\bibitem{3}
	Изюмский А.А., Надирян С.Л. и Сенин И.С. Применение имитационного моделирования в сфере моделирования транспортных потоков. [Электронный ресурс] $//$ Наука. Техника. Технологии (Политехнический вестник). -- 2016. -- Номер 1. -- С.52-54. -- URL: $https://elibrary.ru/item.asp?id=25897722$ (Дата обращения: 09.12.2021)
	\bibitem{4}
	Феофилова А.А., Фиалкин В.В. Решение задач распределения транспортных потоков на основе моделирования дорожного движения на микро-, мезо-, макроуровнях. [Электронный ресурс] $//$ Статья в сборнике трудов конференции <<Строительство и архитектура -- 2015>>. -- 2015. -- С.41-44. -- URL: $https://elibrary.ru/item.asp?id=24981305$ (Дата обращения: 09.12.2021)
	\bibitem{5}
	Изотов А.И. Моделирование транспортных потоков с помощью клеточного автомата. [Электронный ресурс] $//$ Заметки по информатике и математике. -- 2017. -- С.68-75. -- URL: $https://elibrary.ru/item.asp?id=30078073$ (Дата обращения: 09.12.2021)
	\bibitem{6}
	Архангельский А.Н. Моделирование движения автомобиля в транспортном потоке на ЭВМ. [Электронный ресурс] $//$ Автоматизация и моделирование в проектировании и управлении. -- 2020. -- Номер 1(7). -- С.26-31. -- URL: $https://elibrary.ru/item.asp?id=42543955$ (Дата обращения: 09.12.2021)
	\bibitem{7}
	Sanborn. What We Do. 3D Visualization. 3D Cities. [Электронный ресурс] -- URL: $http://www.sanborn.com/3d-cities/$ (Дата обращения: 11.12.2020)
	\bibitem{8}
	OpenStreetMap. [Электронный ресурс] -- URL: $https://www.openstreetmap.org/search?\\query=\%D0\%98\%D0\%B2\%D0\%B0\%D0\%BD\%D1\%82\%D0\\\%B5\%D0\%B5\%D0\%B2\%D0\%BA\%D0\%B0\#map=14/55.9717/37.9305$ (Дата обращения: 05.12.2020)
	\bibitem{9}
	ViziCities. [Электронный ресурс] -- URL: $https://github.com/UDST/vizicities$ (Дата обращения: 12.11.2020)
	\bibitem{10}
	AUTODESK. [Электронный ресурс] -- URL: $https://www.autodesk.com$ (Дата обращения: 11.11.2020)
	\bibitem{11}
	ArcGIS online. [Электронный ресурс] -- URL: $https://www.arcgis.com/index.html$ (Дата обращения: 12.10.2020)
	\bibitem{12}
	Обзор навигационных систем и программ, которыми я пользуюсь. [Электронный ресурс] -- URL: $http://www.encyclopedia-stranstviy.com/2014/05/programmy-prosmotr-kart-navigatsiya.html$ (Дата обращения: 22.12.2020)
	\bibitem{13}
	Magellan VantagePoint™ (Версии 2.27-2.43). Неофициальная инструкция пользователю. [Электронный ресурс] -- URL: $http://www.hllab.dp.ua/Tour/explorist/vpumru.htm$ (Дата обращения: 22.12.2020)
	\bibitem{14}
	Бурлуцкая А.Г., Локтионова Т.С. Обзор технических средств регулирования дорожного движения -- дорожных знаков. [Электронный ресурс] $//$ Статья в сборнике трудов конференции <<Образование, наука, производство>>. -- 2015. -- С.942-944. -- URL: $https://elibrary.ru/item.asp?id=25571569$ (Дата обращения: 02.11.2020)
	\bibitem{15}
	Конвенция о Дорожных Знаках и Сигналах 1968 года. Европейское Соглашение, дополняющее Конвенцию, и Протокол о разметке дорог к Европейскому Соглашению (Сводный текст 2006 года). -- Организация Объединённых Наций. -- Нью-Йорк и Женева. -- 2007 год. -- 239 с. -- URL: $http://www.unece.org/fileadmin/DAM/trans/conventn/Conv\_road\_\\
	signs\_2006v\_RU.pdf$ (Дата обращения: 29.11.2020)
	\bibitem{16}
	Кормилицына Л.В., Дуров Г.Р. Влияние дорожных знаков и разметки на безопасность дорожного движения. [Электронный ресурс] $//$ Дальний Восток. Автомобильные дороги и безопасность движения. -- 2017. -- С.188-191. -- URL: $https://elibrary.ru/item.asp?id=34948281$ (Дата обращения: 04.11.2020)
	\bibitem{17}
	Дорожная разметка РФ. [Электронный ресурс] -- URL: $https://pdd-russia.com/pdd-russia/znaki-i-razmetka/dorozhnaja-razmetka/russia.html$ (Дата обращения: 05.11.2020)
	\bibitem{18}
	Карманов Д.С., Марилов В.С. Современное состояние и перспективы развития платных дорог в России. [Электронный ресурс] $//$ Развитие теории и практики автомобильных перевозок, транспортной логистики. -- 2017. -- С.136-140. -- URL: $https://elibrary.ru/item.asp?id=32492205$ (Дата обращения: 24.10.2020)
	\bibitem{19}
	Алексеева Е.Ю. Обоснование размещения пунктов сбора платежей при проектировании платных автомобильных дорог. [Электронный ресурс] $//$ Дальний Восток. Автомобильные дороги и безопасность движения. -- 2016. -- С.184-187. -- URL: $https://elibrary.ru/item.asp?id=29023954$ (Дата обращения: 04.12.2020)
	\bibitem{20}
	Проезд по М-1. [Электронный ресурс] -- URL: $https://m-road.ru/road/$ (Дата обращения: 22.10.2020)
	\bibitem{21}
	Nikolaos Tsanakas, Joakim Ekström and Johan Olstam. Estimating Emissions from Static Traffic Models: Problems and Solutions. [Text] $//$ Journal of Advanced Transportation. -- 2020. -- Volume 2020. -- 17 pages. -- URL: DOI $10.1155/2020/5401792$ (Дата обращения: 13.03.2020)
	\bibitem{22}
	Fang Zong, Meng Zeng, Wei Zhong and Fengrui Lu. Hybrid Path Selection Modeling by Considering Habits and Traffic Conditions. [Text] $//$ IEEE Access. -- 2019. -- Volume 7. -- Pages 43781-43794. -- URL: DOI $10.1109/ACCESS.2019.2907725$ (Дата обращения: 13.09.2020)
	\bibitem{23}
	MD Jahedul Alam, Muhammad Ahsanul Habib. Mass Evacuation of Halifax, Canada: A Dynamic Traffic Microsimulation Modeling Approach. [Text] $//$ Procedia Computer Science. -- 2019. -- Volume 151. -- Pages 535-542. -- URL: DOI $10.1016/j.procs.2019.04.072$ (Дата обращения: 13.09.2020)
	\bibitem{24}
	Aimsun. [Электронный ресурс] -- URL: $https://www.aimsun.com/$ (Дата обращения: 17.09.2020)
	\bibitem{25}
	McTrans Center, University of Florida. TSIS-CORSIM. Overview. [Электронный ресурс] -- URL: $https://mctrans.ce.ufl.edu/mct/index.php/tsis-corsim/$ (Дата обращения: 17.09.2020)
	\bibitem{26}
	Caliper Corporation. TransModeler Traffic Simulation Software. [Электронный ресурс] -- URL: $https://www.caliper.com/transmodeler/default.htm$ (Дата обращения: 12.09.2020)
	\bibitem{27}
	Имитационное моделирование в проектах ИТС: учебное пособие [Текст] / С.В. Жанказиев, А.И. Воробьев, А.В. Шадрин, М.В. Гаврилюк; под ред.д-ра техн. наук, проф. С.В. Жанказиева. --М.: МАДИ, 2016. -- 92 с.
	\bibitem{28}
	Дмитрий Беспалов. PTV VISSIM: моделирование транспортных потоков. [Электронный ресурс] -- URL: $https://bespalov.me/2012/12/03/ptv-vissim-modelirovanie-transportnih-potokov/$ (Дата обращения: 26.11.2020)
	\bibitem{29}
	Транспортное планирование. PTV VISUM. PTV VISSIM. $//$ URL: $https://ptv-vision.ru/$ (Дата обращения: 27.11.2020)
	\bibitem{30}
	База решений и правовых актов. Решение б/н Решение и предписание по делу № 10-07/16 в отношении Управле... от 13 июля 2017 г. -- URL: $https://br.fas.gov.ru/to/chelyabinskoe-ufas-rossii/10-07-16-f2a89efe-4078-41a1-a20f-36a0e9ed0886/$ (Дата обращения: 22.12.2020)
	\bibitem{31}
	Qiyuan Liu, Jian Sun, Ye Tian and Lu Xiong. Modeling and simulation of overtaking events by heterogeneous non-motorized vehicles on shared roadway segments. [Text] $//$ Simulation Modelling Practice and Theory. -- 2020. -- Volume 103. -- URL: DOI $10.1016/j.simpat.2020.102072$ (Дата обращения: 12.10.2020)
	\bibitem{32}
	Shuichao Zhang, Gang Ren and Renfa Yang. Simulation model of speed-density characteristics for mixed bicycle flow-Comparison between cellular automata model and gas dynamics model. [Text] $//$ Phisica A: Statistical Mechanics and its Applications. -- 2013. -- Volume 392. -- Issue 20. Pages 5110-5118. -- URL: DOI $10.1016/j.physa.2013.06.019$ (Дата обращения 17.10.2020)
	\bibitem{33}
	Sarosh I. Khan, Pawan Maini. Modeling Heterogeneous Traffic Flow. [Text] $//$ Transportation Research Record. -- 1999. -- Pages 234-241. -- URL: DOI $10.3141\%2F1678-28$ (Дата обращения: 14.10.2020)
	\bibitem{34}
	Jinxing Shen, Junje Qi, Feng Qiu and Changjang Zheng. Simulation of Road Capacity Considering the Influence of Buses. [Text] $//$ IEEE Access. -- 2019. -- Volume 7. -- Pages 144178-144187. -- URL: DOI $10.1109/ACCESS.2019.2942524$ (Дата обращения: 13.10.2020)
	\bibitem{35}
	Котов В.Е. Сети Петри. [Текст] / В.Е. Котов -- М. Наука. Главная редакция физико-математической литературы, 1984. -- 160 с.
	\bibitem{36}
	Калпун Н.В., Мирза Н.С. Модель транспортной сети для решения задач моделирования транспортных потоков с использованием сетей Петри. [Электронный ресурс] $//$ Вестник Томского государственного университета. -- 2006. -- Номер $S19$. -- С.243-249. -- URL: $https://www.elibrary.ru/item.asp?id=35182748$ (Дата обращения: 12.11.2020)
	\bibitem{37}
	Графовые базы данных: святой Грааль для разработчиков? [Электронный ресурс] -- URL: $https://habr.com/ru/post/274383/$ (Дата обращения: 20.05.2021)
	\bibitem{38}
	InfiniteGraph. [Электронный ресурс] -- URL: $https://infinitegraph.com/$ (Дата обращения: 20.05.2021)
	\bibitem{39}
	Welcome to Apache Giraph! [Электронный ресурс] -- URL: $http://giraph.apache.org/$ (Дата обращения: 20.05.2021)
	\bibitem{40}
	JanusGraph. [Электронный ресурс] -- URL: $https://janusgraph.org/$ (Дата обращения: 20.05.2021)
	\bibitem{41}
	neo4j 4.2.1. [Электронный ресурс] -- URL: $https://pypi.org/project/neo4j/4.2.1/$ (Дата обращения: 18.05.2021)
	\bibitem{42}
	Сергей Тимофеев. IDEF0-стандарт. [Электронный ресурс] -- URL: $https://itstan.ru/funk-strukt-analiz/idef0-standart.html$ (Дата обращения: 28.11.2021)
	\bibitem{43}
	Дейт К.Дж. Введение в системы баз данных. [Текст] $//$ К.Дж. Дейт. -- 8-е издание. -- 2005. -- Часть 1. Основные понятия. -- С.46.
	\bibitem{44}
	Крёнке Д. Теория и практика построения баз данных. [Текст] $//$ Д. Крёнке. -- 8-е издание. -- 2003. -- Часть II. Моделирование данных. Гл. 5. Реляционная модель и нормализация. -- С.166-201.
	\bibitem{45}
	Шустова И.Б. Данные, хранимые в виде графов. Области применения, перспективы, проблемы манипуляции. [Электронный ресурс] $//$ Статья в сборниках трудов конференции <<Альманах научных работ молодых учёных университета ИТМО>>. -- 2017. -- С.274-277. -- URL: $https://elibrary.ru/item.asp?id=35403998$ (Дата обращения: 21.09.2020)
	\bibitem{46}
	Календарев А. NoSQL как он есть. [Электронный ресурс] $//$ Системный администратор. -- 2013. -- Номер 11 (132). -- С.51-55. -- URL: $https://elibrary.ru/item.asp?id=20466327$ (Дата обращения: 09.09.2020)
	\bibitem{47}
	Ткаченко А.В., Васильчикова А.В., Гришунов С.С. Обзор классов нереляционных баз данных. [Электронный ресурс] $//$ Электронный журнал: Наука, техника и образование. -- 2016. -- Номер 4 (9). -- С.81-85. -- URL: $https://elibrary.ru/item.asp?id=27664308$ (Дата обращения: 10.09.2020)
	\bibitem{48}
	Бочкарев П.В., Кононова М.В. Графовые модели данных. [Электронный ресурс] $//$ Теория. Практика. Инновации. -- 2016. -- Номер 12 (12). -- С.133-141. -- URL: $https://elibrary.ru/item.asp?id=27725671$ (Дата обращения: 13.09.2020)
	\bibitem{49}
	Neo4j Graph Platform. [Электронный ресурс] -- URL: $https://neo4j.com/developer/graph-platform/$ (Дата обращения: 06.05.2021)
	\bibitem{50}
	Liza Shkirando. Neo4j Desktop 1.4.0 Release. [Электронный ресурс] -- URL: $https://medium.com/neo4j/neo4j-desktop-1-4-0-release-c50de440c535$ (Дата обращения: 18.05.2021)
	\bibitem{51}
	Neo4j Download Center. [Электронный ресурс] -- URL: $https://neo4j.com/download-center/\#desktop$ (Дата обращения: 18.05.2021)
	\bibitem{52}
	Python. [Электронный ресурс] -- URL: $www.python.org$ (Дата обращения: 06.05.2021)
	\bibitem{53}
	Using Neo4j from Python. [Электронный ресурс] -- URL: $https://neo4j.com/developer/python/$ (Дата обращения: 18.05.2021)
	\bibitem{54}
	neo4j 4.2.1. [Электронный ресурс] -- URL: $https://pypi.org/project/neo4j/4.2.1/$ (Дата обращения: 18.05.2021)
	\bibitem{55}
	Neo4j Python Driver Wiki. [Электронный ресурс] -- URL: $https://github.com/neo4j/neo4j-python-driver/wiki$ (Дата обращения: 19.05.2021)
	\bibitem{56}
	Tomaz Bratanic. Exploring Pathfinding Graph Algorithms. [Электронный ресурс] -- 2020. -- URL: $https://towardsdatascience.com/part-2-exploring-pathfinding-graph-algorithms-e194ffb7f569$ (Дата обращения: 19.05.2021)
	\bibitem{57}
	Neo4j official page. Официальный сайт Neo4j. [Электронный ресурс] -- URL: $https://neo4j.com/$ (Дата обращения: 04.03.2021)
	\bibitem{58}
	Neo4j Desktop. [Электронный ресурс] -- URL: $https://neo4j.com/docs/desktop-manual/current/$ (Дата обращения: 15.03.2021)
	\bibitem{59}
	Cypher Query Language. [Электронный ресурс] -- URL: $https://neo4j.com/developer/cypher/$ (Дата обращения: 09.03.2021)
	\bibitem{60}
	The Py2neo v4 Handbook. [Электронный ресурс] -- URL: $https://py2neo.org/v4/$ (Дата обращения: 05.05.2021)
	\bibitem{61}
	Крестов С.Г., Строганов Ю.В. Проверка времени исполнения сгенерированных запросов к графовой базе данных. [Электронный ресурс] $//$ Новые информационные технологии в автоматизированных системах. -- 2017. -- Номер 20. -- С.235-238. -- URL: $https://elibrary.ru/item.asp?id=29109664$ (Дата обращения: 27.04.2021)
	\bibitem{62}
	Neo4j official page. Официальный сайт Neo4j. [Электронный ресурс] -- URL: $https://neo4j.com/$ (Дата обращения: 04.05.2021)
	\bibitem{63}
	Neo4j official documentation. Официальная документация Neo4j. [Электронный ресурс] -- URL: $https://neo4j.com/docs/operations-manual/current/tools/cypher-shell/$ (Дата обращения: 04.05.2021)
	\bibitem{64}
	Оселедец И.В. Прототипирование программных комплексов. [Электронный ресурс] $//$ Статья в сборнике трудов конференции <<Научный сервис в сети Интернет: поиск новых решений>>. -- 2012. -- С.404-411. -- URL: $https://elibrary.ru/item.asp?id=22447021$ (Дата обращения: 30.04.2021)
	\bibitem{65}
	Docs petri. [Электронный ресурс] -- URL: $https://petri.readthedocs.io/en/latest/$ (Дата обращения: 30.01.2022)
	\bibitem{66}
	petri 0.24.1. [Электронный ресурс] -- URL: $https://pypi.org/project/petri/$ (Дата обращения: 26.01.2022)
	\bibitem{67}
	SNAKES 0.9.29. [Электронный ресурс] -- URL: $https://pypi.org/project/SNAKES/$ (Дата обращения: 07.02.2022)
	\bibitem{68}
	PM4PY. Documentation. Petri Net properties. Creating a new Petri Net. [Электронный ресурс] -- URL: $https://pm4py.fit.fraunhofer.de/documentation$ (Дата обращения: 31.01.2022)
	\bibitem{69}
	D. A. Shibanova, I. V. Stroganov and I. V. Rudakov, "Data Formalization in Transport System Modeling Using a Graph Database," 2021 IEEE Conference of Russian Young Researchers in Electrical and Electronic Engineering (ElConRus), 2021, pp. 2245-2251, doi: $10.1109/ElConRus51938.2021.9396137$.
	\bibitem{70}
	Шибанова Д.А., Строганов Ю.В. Моделирование дорожной системы с использованием графовой базы данных. Политехнический молодежный журнал, 2022, No 01(66). $http://dx.doi.org/10.18698/2541-8009-2022-01-765$ Shibanova D.A., Stroganov Yu.V. Modeling a road system using a graph database. Politekhnicheskiy molodezhnyy zhurnal [Politechnical student journal], 2022, no. 01(66). $http://dx.doi.org/10.18698/2541-8009-2022-01-765.html$ (in Russ.).
	\bibitem{71}
	Estelle Scifo. Intoducing Neomap, a Neo4j Desktop application for spatial data. [Электронный ресурс] -- URL: $https://medium.com/neo4j/introducing-neomap-a-neo4j-desktop-application-for-spatial-data-3e14aad59db2$ (Дата обращения: 16.03.2022)
	\bibitem{72}
	Estelle Scifo. Visualizing shortest paths with neomap $\geq$ 0.4.0 and the Neo4j Graph Data Science plugin. [Электронный ресурс] -- URL: $https://medium.com/neo4j/visualizing-shortest-paths-with-neomap-0-4-0-and-the-neo4j-graph-data-science-plugin-18db92f680de$ (Дата обращения: 16.03.2022)
	\bibitem{73}
	stellasia / neomap. [Электронный ресурс] -- URL: $https://github.com/stellasia/neomap$ (Дата обращения: 14.05.2022)
\end{thebibliography}
%\endgroup